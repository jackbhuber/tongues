% Options for packages loaded elsewhere
\PassOptionsToPackage{unicode}{hyperref}
\PassOptionsToPackage{hyphens}{url}
\PassOptionsToPackage{dvipsnames,svgnames,x11names}{xcolor}
%
\documentclass[
  letterpaper,
  DIV=11,
  numbers=noendperiod]{scrreprt}

\usepackage{amsmath,amssymb}
\usepackage{iftex}
\ifPDFTeX
  \usepackage[T1]{fontenc}
  \usepackage[utf8]{inputenc}
  \usepackage{textcomp} % provide euro and other symbols
\else % if luatex or xetex
  \usepackage{unicode-math}
  \defaultfontfeatures{Scale=MatchLowercase}
  \defaultfontfeatures[\rmfamily]{Ligatures=TeX,Scale=1}
\fi
\usepackage{lmodern}
\ifPDFTeX\else  
    % xetex/luatex font selection
\fi
% Use upquote if available, for straight quotes in verbatim environments
\IfFileExists{upquote.sty}{\usepackage{upquote}}{}
\IfFileExists{microtype.sty}{% use microtype if available
  \usepackage[]{microtype}
  \UseMicrotypeSet[protrusion]{basicmath} % disable protrusion for tt fonts
}{}
\makeatletter
\@ifundefined{KOMAClassName}{% if non-KOMA class
  \IfFileExists{parskip.sty}{%
    \usepackage{parskip}
  }{% else
    \setlength{\parindent}{0pt}
    \setlength{\parskip}{6pt plus 2pt minus 1pt}}
}{% if KOMA class
  \KOMAoptions{parskip=half}}
\makeatother
\usepackage{xcolor}
\setlength{\emergencystretch}{3em} % prevent overfull lines
\setcounter{secnumdepth}{5}
% Make \paragraph and \subparagraph free-standing
\makeatletter
\ifx\paragraph\undefined\else
  \let\oldparagraph\paragraph
  \renewcommand{\paragraph}{
    \@ifstar
      \xxxParagraphStar
      \xxxParagraphNoStar
  }
  \newcommand{\xxxParagraphStar}[1]{\oldparagraph*{#1}\mbox{}}
  \newcommand{\xxxParagraphNoStar}[1]{\oldparagraph{#1}\mbox{}}
\fi
\ifx\subparagraph\undefined\else
  \let\oldsubparagraph\subparagraph
  \renewcommand{\subparagraph}{
    \@ifstar
      \xxxSubParagraphStar
      \xxxSubParagraphNoStar
  }
  \newcommand{\xxxSubParagraphStar}[1]{\oldsubparagraph*{#1}\mbox{}}
  \newcommand{\xxxSubParagraphNoStar}[1]{\oldsubparagraph{#1}\mbox{}}
\fi
\makeatother


\providecommand{\tightlist}{%
  \setlength{\itemsep}{0pt}\setlength{\parskip}{0pt}}\usepackage{longtable,booktabs,array}
\usepackage{calc} % for calculating minipage widths
% Correct order of tables after \paragraph or \subparagraph
\usepackage{etoolbox}
\makeatletter
\patchcmd\longtable{\par}{\if@noskipsec\mbox{}\fi\par}{}{}
\makeatother
% Allow footnotes in longtable head/foot
\IfFileExists{footnotehyper.sty}{\usepackage{footnotehyper}}{\usepackage{footnote}}
\makesavenoteenv{longtable}
\usepackage{graphicx}
\makeatletter
\def\maxwidth{\ifdim\Gin@nat@width>\linewidth\linewidth\else\Gin@nat@width\fi}
\def\maxheight{\ifdim\Gin@nat@height>\textheight\textheight\else\Gin@nat@height\fi}
\makeatother
% Scale images if necessary, so that they will not overflow the page
% margins by default, and it is still possible to overwrite the defaults
% using explicit options in \includegraphics[width, height, ...]{}
\setkeys{Gin}{width=\maxwidth,height=\maxheight,keepaspectratio}
% Set default figure placement to htbp
\makeatletter
\def\fps@figure{htbp}
\makeatother
% definitions for citeproc citations
\NewDocumentCommand\citeproctext{}{}
\NewDocumentCommand\citeproc{mm}{%
  \begingroup\def\citeproctext{#2}\cite{#1}\endgroup}
\makeatletter
 % allow citations to break across lines
 \let\@cite@ofmt\@firstofone
 % avoid brackets around text for \cite:
 \def\@biblabel#1{}
 \def\@cite#1#2{{#1\if@tempswa , #2\fi}}
\makeatother
\newlength{\cslhangindent}
\setlength{\cslhangindent}{1.5em}
\newlength{\csllabelwidth}
\setlength{\csllabelwidth}{3em}
\newenvironment{CSLReferences}[2] % #1 hanging-indent, #2 entry-spacing
 {\begin{list}{}{%
  \setlength{\itemindent}{0pt}
  \setlength{\leftmargin}{0pt}
  \setlength{\parsep}{0pt}
  % turn on hanging indent if param 1 is 1
  \ifodd #1
   \setlength{\leftmargin}{\cslhangindent}
   \setlength{\itemindent}{-1\cslhangindent}
  \fi
  % set entry spacing
  \setlength{\itemsep}{#2\baselineskip}}}
 {\end{list}}
\usepackage{calc}
\newcommand{\CSLBlock}[1]{\hfill\break\parbox[t]{\linewidth}{\strut\ignorespaces#1\strut}}
\newcommand{\CSLLeftMargin}[1]{\parbox[t]{\csllabelwidth}{\strut#1\strut}}
\newcommand{\CSLRightInline}[1]{\parbox[t]{\linewidth - \csllabelwidth}{\strut#1\strut}}
\newcommand{\CSLIndent}[1]{\hspace{\cslhangindent}#1}

\KOMAoption{captions}{tableheading}
\makeatletter
\@ifpackageloaded{bookmark}{}{\usepackage{bookmark}}
\makeatother
\makeatletter
\@ifpackageloaded{caption}{}{\usepackage{caption}}
\AtBeginDocument{%
\ifdefined\contentsname
  \renewcommand*\contentsname{Table of contents}
\else
  \newcommand\contentsname{Table of contents}
\fi
\ifdefined\listfigurename
  \renewcommand*\listfigurename{List of Figures}
\else
  \newcommand\listfigurename{List of Figures}
\fi
\ifdefined\listtablename
  \renewcommand*\listtablename{List of Tables}
\else
  \newcommand\listtablename{List of Tables}
\fi
\ifdefined\figurename
  \renewcommand*\figurename{Figure}
\else
  \newcommand\figurename{Figure}
\fi
\ifdefined\tablename
  \renewcommand*\tablename{Table}
\else
  \newcommand\tablename{Table}
\fi
}
\@ifpackageloaded{float}{}{\usepackage{float}}
\floatstyle{ruled}
\@ifundefined{c@chapter}{\newfloat{codelisting}{h}{lop}}{\newfloat{codelisting}{h}{lop}[chapter]}
\floatname{codelisting}{Listing}
\newcommand*\listoflistings{\listof{codelisting}{List of Listings}}
\makeatother
\makeatletter
\makeatother
\makeatletter
\@ifpackageloaded{caption}{}{\usepackage{caption}}
\@ifpackageloaded{subcaption}{}{\usepackage{subcaption}}
\makeatother

\ifLuaTeX
  \usepackage{selnolig}  % disable illegal ligatures
\fi
\usepackage{bookmark}

\IfFileExists{xurl.sty}{\usepackage{xurl}}{} % add URL line breaks if available
\urlstyle{same} % disable monospaced font for URLs
\hypersetup{
  pdftitle={Effects of Glossolalia},
  pdfauthor={Jack B. Huber, Ph.D.},
  colorlinks=true,
  linkcolor={blue},
  filecolor={Maroon},
  citecolor={Blue},
  urlcolor={Blue},
  pdfcreator={LaTeX via pandoc}}


\title{Effects of Glossolalia}
\author{Jack B. Huber, Ph.D.}
\date{2022-11-16}

\begin{document}
\maketitle

\renewcommand*\contentsname{Table of contents}
{
\hypersetup{linkcolor=}
\setcounter{tocdepth}{2}
\tableofcontents
}

\bookmarksetup{startatroot}

\chapter*{Abstract}\label{abstract}
\addcontentsline{toc}{chapter}{Abstract}

\markboth{Abstract}{Abstract}

Some Christians speak in tongues. This study uses data from the Pew
Research Center 2014 Religious Landscape Study to examine the effect of
glossolalia on a variety of religious, social, and political attitudes
and behavior.

\bookmarksetup{startatroot}

\chapter*{Background}\label{background}
\addcontentsline{toc}{chapter}{Background}

\markboth{Background}{Background}

In the Christian world, glossolalia is the practice of speaking or
praying in ``tongues.'' Glossolalia is particularly associated the
Pentecostal tradition which emphasizes the direct ``infilling'' of the
Holy Spirit. While most people who practice glossolalia tend to see it
as direct evidence of divine presence, not all will claim that lack of
it means lack of divine presence.

\section*{Research Questions}\label{research-questions}
\addcontentsline{toc}{section}{Research Questions}

\markright{Research Questions}

In this study, I use data from a recent national survey to explore the
practice of glossolalia in the population of American adults. I pose the
following research questions:

\begin{enumerate}
\def\labelenumi{\arabic{enumi}.}
\tightlist
\item
  Who speaks in tongues? Which traditions? Which demographics?
\item
  Correlates of glossolalia. What religious variables are most
  predictive of glossolalia? Is there something special about speaking
  in tongues?
\item
  Effects of glossolalia on attitudes, psychology well-being?
\end{enumerate}

\bookmarksetup{startatroot}

\chapter*{Methods}\label{methods}
\addcontentsline{toc}{chapter}{Methods}

\markboth{Methods}{Methods}

\section*{Data Source}\label{data-source}
\addcontentsline{toc}{section}{Data Source}

\markright{Data Source}

The data for this project come from the 2014 Religious Landscape Study,
by the Pew Research Center.

\section*{Instruments}\label{instruments}
\addcontentsline{toc}{section}{Instruments}

\markright{Instruments}

\subsection*{Outcome variables}\label{outcome-variables}
\addcontentsline{toc}{subsection}{Outcome variables}

\subsection*{Predictor variables}\label{predictor-variables}
\addcontentsline{toc}{subsection}{Predictor variables}

\textbf{Frequency of praying or speaking in tongues.} Measurement of
glossolalia comes from one item in the Religious Landscape Survey.
Respondents how identified with one of the following Christian
traditions -- Evangelical Protestant Tradition, Mainline Protestant
Tradition, Historically Black Protestant Tradition, Catholic, Mormon,
Orthodox Christian, Jehovah's Witness, Other Christian -- were asked how
often they ``speak or pray in tongues'' with the following response
options: at least once a week, once or twice a month, several times a
year, seldom, or never.

\section*{Statistical Analysis}\label{statistical-analysis}
\addcontentsline{toc}{section}{Statistical Analysis}

\markright{Statistical Analysis}

To investigate the prevalance of glossolalia in the Christian world I
used basic descriptive statistics. To isolate the effect of glossolalia
on attitudes I used OLS regression.

\bookmarksetup{startatroot}

\chapter*{Results}\label{results}
\addcontentsline{toc}{chapter}{Results}

\markboth{Results}{Results}

\section*{Prevalence of glossolalia}\label{prevalence-of-glossolalia}
\addcontentsline{toc}{section}{Prevalence of glossolalia}

\markright{Prevalence of glossolalia}

Table 1 presents the frequencies of responses to the item asking
respondents how often they pray or speak in tongues.

X percent of self-identified American Christians, estimate XX,XXX
people, engage in the practice. Thus glossolalia does not appear to be a
prevalent practice.

\section*{Predictors of Glossolalia}\label{predictors-of-glossolalia}
\addcontentsline{toc}{section}{Predictors of Glossolalia}

\markright{Predictors of Glossolalia}

\section*{Effects of Glossolalia}\label{effects-of-glossolalia}
\addcontentsline{toc}{section}{Effects of Glossolalia}

\markright{Effects of Glossolalia}

\bookmarksetup{startatroot}

\chapter*{Discussion}\label{discussion}
\addcontentsline{toc}{chapter}{Discussion}

\markboth{Discussion}{Discussion}

\bookmarksetup{startatroot}

\chapter*{References}\label{references}
\addcontentsline{toc}{chapter}{References}

\markboth{References}{References}

\phantomsection\label{refs}
\begin{CSLReferences}{0}{1}
\end{CSLReferences}




\end{document}
